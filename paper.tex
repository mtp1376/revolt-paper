\documentclass{article}
\usepackage{xepersian}
\author{محمدرضا خان‌محمدی و محمد تیموری‌پابندی}
\title{بررسی مقاله‌ی «ایران و استعمار سرخ و سیاه»}
\begin{document}
\maketitle
\textit{ایران و استعمار سرخ و سیاه}
نام مقاله‌ای بود که در 17 دی 1356 در روزنامه‌ی اطلاعات با امضای
\textit{احمد رشیدی مطلق}
منتشر شد. این مقاله که با نام مستعار نوشته شده بود، حاوی مطالبی تند علیه انقلاب اسلامی بود. این مقاله‌ی توهین‌آمیز، با اعتراضات زیادی روبرو شد و نقطه‌ی عطفی در شتاب‌گرفتن ناگهانی اعتراضات به حکومت پهلوی شد.
\par
این مقاله، بی‌ربط با سفر جیمی کارتر، 38 امین رئیس جمهور دموکرات آمریکا، به ایران در سال 1356 نبود و اصلاً در همان روزهای سفر کارتر و هیأت همراه 410نفره‌اش به تهران در کریسمس 1978 منتشر شد؛ سفر 17ساعته‌ای که حاشیه‌های بسیار زیادی به همراه داشت و چندان بی‌تأثیر بر پایان عمر حکومت نیم‌قرنه‌ی پهلوی نبود.
\par
کارتر در این مراسم، ایران را «جزیره‌ی ثبات
\پانویس{Island of Stability}
» در یکی بحران‌زده‌ترین مناطق دنیا یعنی خاورمیانه خواند، خاورمیانه‌ای که سال‌ها شاهد ستیز اعراب با دشمن دیرینه‌ی فلسطینیان، یعنی رژیم صهیونیستی، که تازه‌ترین درگیریشان در آن روز‌ها «یوم کیپور» یا نبرد اصطکاک نام داشت که به 1973 برمی‌گشت بود.
\par
و خاورمیانه‌ای که شاهد تنش‌های ممتد در روزهای به قدرت رسیدن صدام حسین، این مجنون معروف به دیکتاتور کیش در عراق بود.
\par
این سخنان از زبان کسی شنیده می‌شد که به گفته‌ی محمدرضا پهلوی، نمی‌توانست همزمان هم آدامس بجود و هم حرف بزند.
\par
پی‌یر سالینجر
\پانویس{Pierre Emil George Salinger}
در سال 1981 در کتابی با نام «آمریکا گروگان را حفظ می‌کند
\پانویس{America Held Hostage: The Secret Negotiations, 1981}
» می‌نویسد که دیپلمات‌های آمریکایی حاضر در این محل از سخنان کارتر «شگفت‌زده» شده‌اند؛ زیرا جیمی کارتر در مبارزات انتخاباتی خود صریحاً به سیاست حقوق بشر تأکید کرده بود و همزمان ایران توسط عفو بین‌الملل به دلیل موارد نقض حقوق بشر محکوم شده بود. حتی در صورتی هم که کارتر از گزارش عفو بین‌الملل آگاه نمی‌بود، حتماً از گزارش‌های سفارت آمریکا در تهران در مورد شکنجه‌ی مخالفان شاه مطلع بود.
\par
شاه پس از این سخنان حمایت‌آمیز، قدرت مضاعفی را برای مقابله با مخالفانش یافت. یک‌هفته بعد از سفر کارتر یعنی در تاریخ 17 دی 1356، مقاله‌ی جنجالی \textit{«ایران و استعمار سرخ و سیاه»} در روزنامه‌ی اطلاعات علیه انقلاب اسلامی به چاپ رسید. در واکنش به این مقاله، ناآرامی‌هایی در کشور رخ داد. در مشهد تظاهرات انجام شد و در روز 18دی در قم بازار و حوزه‌ی علمیه تعطیل شدند.
\par
تظاهرات مردم قم با تیراندازی نیروهای حکومت کشته داد. دانشجویان دانشکده‌ی فنی تهران نیز 2روز بعد در محوطه‌ی دانشگاه تظاهرات کردند و کلاس‌ها را تعطیل کردند و بازار تهران نیز در روز چهلم شهادت کشتگان قم تعطیل شد. بدین ترتیب سلسله راهپیمایی‌های روز چهلم شهدای شهرها آغاز شد و هر 40روز در شهری تظاهرات می‌شد.
\par
درج مقاله‌ی هتاکانه‌ی «ایران و استعمار سرخ و سیاه» در صفحه‌ی 7 روزنامه‌ی اطلاعات، مورخ 17 دی‌ماه 1356، نقطه‌ی آغازی بر پایان رژیم شاهنشاهی ایران بود. این مقاله که با نام مستعار احمد رشیدی مطلق منتشر شد و از داریوش همایون، وزیر اطلاعات و جهانگردی وقت، نشئت می‌گرفت، چنان طوفانی به راه انداخت که 400روز بعد طومار خاندان پهلوی را در هم پیچید.
\par
در ابتدا هویت واقعی نویسنده‌ی این مقاله‌ی موهن مشخص نبود و امروزه نیز حرف و حدیث‌های زیادی در مورد آن زده می‌شود. اما همان روزهای پس از انتشار مقاله‌ی مذكور هويت واقعی نويسنده‌ی آن در هاله‌ای از ابهام افتاد و صاحب‌نظران برای پی بردن به شخصيت واقعی « احمد رشيدی مطلق» كه نويسنده‌ی مقاله‌ی عنوان شده بود، افراد متعددی را در مظان اتهام قرار دادند. در اين ميان داريوش همايون وزير اطلاعات وقت بيش از ديگران متهم بود كه اين مقاله را تدوين كرده و جهت انتشار به روزنامه اطلاعات سپرده است. اما او كماكان اين اتهام را رد كرده است و ضمن اينكه تصريح می‌كند اين مقاله در پاكتی مهر و موم (كه به آرم وزارت دربار مزين بود) به او تسليم شده و او بدون كنكاش در محتوای آن برای چاپ در اختيار مسئولان روزنامه اطلاعات قرار داده است. قراين و شواهدی هم وجود دارد كه نشان می‌دهد داريوش همايون نمی‌تواند نويسنده‌ی مقاله‌ی مزبور باشد. در اين ميان به ويژه شيوه‌ی نگارش همايون با متن مقاله استعمار سرخ و سياه هيچ‌گونه همخوانی ندارد.
\par
اردشير زاهدی كه در آن روزگار سفير ايران در واشنگتن بود بعدها در خاطراتش تصريح كرد كه آن مقاله‌ی سرنوشت‌ساز را فرهاد نيكوخواه معاون و مشاور مطبوعاتی هويدا در وزارت دربار تدوين كرده بود و اضافه می‌كند كه اندكی پس از آن واقعه شخص فرهاد نيكوخواه را برای اين اقدام نابخردانه شماتت كرده است.
\par
عبدالرضا هوشنگ مهدوی که از دیپلمات های هوادار نهضت ملی بود هم در بخشی از خاطرات خود درباره علت و منشأ تدوين و انتشار مقاله‌ی استعمار سرخ و سياه چنين اظهار نظر كرده است:
\begin{quote}
«شاه دريافت كه چه خطر هولناکی تخت و تاج او را تهديد می‌کند. لذا به فرهاد نيكوخواه مشاور مطبوعاتی هويدا كه اكنون در دربار انجام وظيفه می‌كرد دستور داد مقاله‌ی مشهور استعمار سرخ و سياه را با امضای مستعار بنويسد و در آن به آيت‌الله خمينی حمله كند. پس از پيروزی انقلاب بسياری ادعا كردند كه اين مقاله را داريوش همايون وزير اطلاعات و جهانگردی نوشته است در حالی كه نگارنده [مهدوی] كه نظير اين ماجرا را به چشم ديده بود بدون اينكه احساسات موافقی نسبت به همايون داشته باشد يقين داشت كه همان طور كه وی ادعا می‌کرد پاكت محتوی مقاله را كه از وزارت دربار واصل شده بود دربسته و نخوانده به خبرنگار اطلاعات داده بوده است.»
\end{quote}
\قسمت*{شایعه‌ی تحریم روزنامه‌ی اطلاعات}
\begin{itemize}
\item
در پی درج مقاله‌ی «ايران و استعمار سرخ و سياه» در روزنامه اطلاعات مورخ 2536/10/17، بين بازاريان و كسبه تهران شايعه گرديده عده‌ای از روحانيون خريد روزنامه مذكور را تحريم نموده‌اند.
\item
پس از درج مقاله در روزنامه اطلاعات در مورد امام خمينی سطح فروش روزنامه‌ی مزبور در سطح كشور كاهش يافته و در آبان هم روزانه حدود 300 شماره كمتر از قبل روزنامه به فروش می‌رسد كه از طريق نمايندگی به تهران عودت داده می‌شود و چندی پيش هم چند نفر از روزنامه‌فروشان آبادان اظهار داشته بودند بنا به فتوای روحانيون روزنامه اطلاعات نجس است و حاضر به قبول روزنامه نبوده‌اند.
\item
نظریه‌ی شنبه: عدم فروش 300شماره‌ی روزنامه نسبت به سابق و عدم قبول روزنامه توسط چند نفر از روزنامه فروشان مورد تأیید می‌باشد.
\item
عده‌ی زیادی از جمله ارتشبد عباس قره‌باغی، ویلیام لوئیس، نیکی کدی و... نگارش این مقاله را، که با تیتری سرخ‌رنگ در شماره‌ی 15506 روزنامه‌ی اطلاعات به چاپ رسیده بود، به داریوش همایون نسبت می‌دهند؛ اما خود داریوش همایون، شاه را مسئول اصلی آماده‌سازی آن مقاله می‌داند و می‌گوید آن مقاله به دستور شاه و به وسیله‌ی دفتر مطبوعاتی هویدا تهیه شده بود و او تنها مجری اوامر بوده است.
\end{itemize}

\قسمت*{علت نگارش مقاله}
برافروخته شدن آتش خشم شاه و تلاش جهت انتقام‌جویی یکی از دلایل مهم چاپ این مقاله بود. پخش شدن نوارهای ممنوعه‌ی سخنرانی امام خمینی در نجف، برای حاضرین در مساجد ایران و روش سرسختانه‌ای که ایشان در مخالفت با شاه در پیش گرفته بودند، به علما و مردم ایران روحیه و شجاعت مضاعفی در مخالفت و خصومت با شاه می‌بخشید.
\par
مرگ ناگهانی حاج‌آقا مصطفی، فرزند امام، که در این زمان در عراق اتفاق افتاد و بسیاری آن را کار ساواک می‌پنداشتند، بر مخالفت‌ها علیه شاه افزود. سیلی از نامه‌های تسلیت از سوی شخصیت‌های ملی و مذهبی داخلی و خارجی برای امام فرستاده شد. یکی از این شخصیت‌ها یاسر عرفات بود که امام در پاسخ نامه‌ی تسلیت‌آمیز وی، در 13 آبان‌ 1356، یادآور شد که عزای من فقط بابت مصطفی نیست. در ادامه ایشان پایان عزایش را منوط به نابودی رژیم پهلوی عنوان کرد.
\par
دلیل دیگری که می‌توان در خصوص انتشار این مقاله به آن اشاره کرد این است که با روی کار آمدن کارتر از حزب دمکرات آمریکا، شاه ایران تحت فشار زیادی برای اعمال فضای باز سیاسی در کشور قرار گرفته بود و برای کاهش این فشارها، در صدد هشداری به مقامات آمریکایی بود. لذا از به حرکت درآوردن مخالفان مذهبی در این راستا بهره برد. بدین ترتیب، درست 8 روز پس از عزیمت کارتر از ایران، مقاله‌ی احمد رشیدی مطلق به چاپ رسید.
\par
از سوی دیگر، ماهیت و ساختار قدرت پهلوی، خود مؤید سخنان داریوش همایون است، چرا که در حکومت استبدادی محمدرضاشاه تمام قدرت در دست شاه بود و انجام هیچ کاری بدون اراده‌ی او امکان‌پذیر نبود و انتشار چنین مقاله‌ی توهین‌آمیزی به مرجع عالی‌قدر جهان تشیع نمی‌توانست بدون اطلاع و بعضاً دستور شخص وی صورت گیرد.
\par
با توجه به آنچه گفته شد، بدیهی است که مقاله‌ی رشیدی مطلق بنا به دستور محمدرضاشاه و با آگاهی کامل وی تهیه و منتشر شد، اما در نزد مردم و اکثر نویسندگان، از داریوش همایون به عنوان احمد رشیدی مطلق واقعی یاد شد.
\par
با بررسی متن مقاله‌ی «ایران و استعمار سرخ و سیاه» اهداف نویسندگان آن به خوبی آشکار می‌شود. \\
نویسنده‌ی مقاله به طور حساب‌شده‌ای از بار معنایی کلمات در عنوان مقاله برای این هدف خود استفاده می‌نماید؛ بدین معنا که از بار منفی کلمه‌ی سیاه (اشاره به روحانیت مبارز) استفاده می‌کند و آن را در مقابل بار مثبت معنایی واژه‌های سفید (انقلاب سفید شاه) قرار می‌دهد تا نه تنها مخالفت‌های امام با انقلاب سفید، بلکه مخالفت امام با رژیم پهلوی را اقدامی منفی و انقلاب سفید و اقدامات شاه را امری مثبت نشان دهد. از سوی دیگر، از واژه‌ی استعمار سرخ استفاده شده است تا نهضت امام خمینی را هم‌سطح و ملهم از لنین و مبارزات کمونیست‌ها قرار دهد.
\قسمت*{پیامدهای این عمل موهن}
انتشار مقاله‌ی هتاکانه‌ی رشیدی مطلق علیه امام خمینی، عکس‌العمل شدیدی را در سراسر ایران به دنبال داشت. اولین بازتاب انتشار این مقاله در مرکز انقلاب، یعنی قم، منعکس گردید. در روز انتشار مقاله، هنگامی که کامیون حامل روزنامه‌های اطلاعات برای قم و اصفهان از دروازه‌ی قم وارد این شهر می‌شد، مردم قم که از پیش خبر انتشار مقاله را از تهران شنیده بودند، به کامیون هجوم بردند و روزنامه‌ها را به آتش کشیدند.
\par
روز بعد طلاب کلاس‌های درس را تعطیل کردند و در مسجد اعظم تحصن نمودند و زد و خوردهای پراکنده‌ای رخ داد، ولی حرکت اصلی مردم در روز 19 دی شکل گرفت. مردم ضمن سر دادن شعارهای تند علیه دولت، نمایندگی روزنامه‌ی اطلاعات و دفتر حزب رستاخیز را آتش زدند. \\ عکس‌العمل رژیم در مقابل مردم وحشیانه بود و نیروهای نظامی به روی مردم آتش گشودند. جمعی را کشتند و عده‌ی زیادی را مجروح و جمعی از روحانیون را نیز دستگیر و تبعید نمودند. بدین ترتیب، 19 دی‌ماه 1356 به خرداد 1342 پیوند خورد و استمرار نهضت امام خمینی را نشان داد.
\par
با این اقدام رژیم در به خاک و خون کشیدن مردم قم، موجی از خشم و نفرت سراسر ایران را فراگرفت. در شهرهای مختلف مراسم بزرگداشت شهدای قم برگزار شد؛ ضمن آنکه امام خمینی نیز خطاب به ملت ایران پیامی فرستاد که موجب گسترش دامنه‌ی مبارزه علیه رژیم شد. در چهلم شهدای قم، در 29 بهمن 1356، مردم تبریز قیام کردند و حرکت نهضت را تداوم بخشیدند. مردم با به آتش کشیدن سینماهایی که فیلم‌های مستهجن را به نمایش می‌گذاشتند، مشروب‌فروشی‌ها و دفتر حزب رستاخیز، خشم خود را نسبت به رژیم ابراز کردند. سرانجام با مداخله‌ی ارتش صحنه‌ی خونینی به وجود آمد و حادثه‌ی قم بار دیگر در تبریز تکرار شد و بیمارستان‌ها مملو از مجروحین و کشته‌شدگان گردید.
\par
\begin{quote}
«...اهالی معظم و عزیز آذربایجان بدانند که در این راه حق و استقلال و آزادی‌طلبی و در حمایت از قرآن کریم تنها نیستند... دیگر شهرها... با آن‌ها هم‌صدا و هم‌مقصد و همه و همه در بیزاری از دودمان پلید پهلوی شریک‌اند.» 
\end{quote}
بدین ترتیب آتش انقلاب که با جرقه‌ی نامه‌ی رشیدی مطلق و توهین به امام خمینی شعله‌ور شده بود، از شهری به شهر دیگر منتقل می‌شد و خرمن رژیم پهلوی را به آتش می‌کشید.
\par
روزهای نهم و دهم فروردین 1357 شعله‌ی انقلاب به یزد رسید و مراسم اربعین شهدای تبریز در این شهر برگزار شد. مردم با شعارهای «الله‌اکبر»، «لا اله الا الله»، «درود بر خمینی» و «مرگ بر شاه» به خیابان‌ها ریختند و حوادث خونین قم و تبریز در یزد نیز تکرار شد و پس از آن، درگیری‌ها و تظاهرات شهر به شهر گسترش یافت. در اصفهان و شهرضا حکومت نظامی اعلام شد و در مرداد 1357 سینما رکس آبادان در یک جنایت مهیب به آتش کشیده شد و بیش از 400 نفر در آتش سوختند. مردم ساواک را عامل آتش‌سوزی می‌دانستند و سهل‌انگاری مسئولین رژیم در پیگیری این مسئله مهر تأییدی بر این باور مردم زد وبر خشم و نفرت مردم از رژیم افزود. 

\begin{thebibliography}{9}

\bibitem{}
پایگاه جامع تاریخ معاصر ایران، به نشانی
\lr{pchi.ir}

\bibitem{}
پورتال جامع علوم انسانی، به نشانی 
\lr{ensani.ir}

\bibitem{}
ویکی‌پدیا، دانشنامه‌ی آزاد، به نشانی
\lr{fa.wikipedia.org}

\bibitem{}
خبرگزاری فارس، به نشانی
\lr{farsnews.com}

\bibitem{}
فریدون هویدا، سقوط شاه، ترجمه‌ی حسین ابوترابیان(تهران: اطلاعات، 1388)
\bibitem{}
ارواند آبراهامیان، ایران بین دو انقلاب: درآمدی بر جامعه‌شناسی سیاسی ایران معاصر، ترجمه‌ی احمد گل‌محمدی، محمدابراهیم فتاحی ولیالی(تهران: نی، 1392)
\bibitem{}
امام خمینی، صحیفه‌ی نور(سازمان چاپ و انتشارات وزارت فرهنگ و ارشاد اسلامی، 1369)

\end{thebibliography}

\end{document}