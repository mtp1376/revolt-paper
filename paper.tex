\documentclass{article}
\usepackage{xepersian}
\author{محمدرضا خان‌محمدی و محمد تیموری‌پابندی}
\title{بررسی مقاله‌ی «ایران و استعمار سرخ و سیاه»}
\begin{document}
\maketitle
\textit{ایران و استعمار سرخ و سیاه}
نام مقاله‌ای بود که در 17 دی 1356 در روزنامه‌ی اطلاعات با امضای
\textit{احمد رشیدی مطلق}
منتشر شد. این مقاله که با نام مستعار نوشته شده بود، حاوی مطالبی تند علیه انقلاب اسلامی بود. این مقاله‌ی توهین‌آمیز، با اعتراضات زیادی روبرو شد و نقطه‌ی عطفی در شتاب‌گرفتن ناگهانی اعتراضات به حکومت پهلوی شد.
\par
این مقاله، بی‌ربط با سفر جیمی کارتر، 38 امین رئیس جمهور دموکرات آمریکا، به ایران در سال 1356 نبود و اصلاً در همان روزهای سفر کارتر و هیأت همراه 410نفره‌اش به تهران در کریسمس 1978 منتشر شد؛ سفر 17ساعته‌ای که حاشیه‌های بسیار زیادی به همراه داشت و چندان بی‌تأثیر بر پایان عمر حکومت نیم‌قرنه‌ی پهلوی نبود.
\par
کارتر در این مراسم، ایران را «جزیره‌ی ثبات»
\پانویس{Island of Stability}
در یکی بحران‌زده‌ترین مناطق دنیا یعنی خاورمیانه خواند، خاورمیانه‌ای که سال‌ها شاهد ستیز اعراب با دشمن دیرینه‌ی فلسطینیان، یعنی رژیم صهیونیستی، که تازه‌ترین درگیریشان در آن روز‌ها «یوم کیپور» یا نبرد اصطکاک نام داشت که به 1973 برمی‌گشت بود.
\par
و خاورمیانه‌ای که شاهد تنش‌های ممتد در روزهای به قدرت رسیدن صدام حسین، این مجنون معروف به دیکتاتور کیش در عراق بود.
\par
این سخنان از زبان کسی شنیده می‌شد که به گفته‌ی محمدرضا پهلوی، نمی‌توانست همزمان هم آدامس بجود و هم حرف بزند.
\par
پی‌یر سالینجر
\پانویس{Pierre Emil George Salinger}
در سال 1981 در کتابی با نام «آمریکا گروگان را حفظ می‌کند»
\پانویس{America Held Hostage: The Secret Negotiations, 1981}
می‌نویسد که دیپلمات‌های آمریکایی حاضر در این محل از سخنان کارتر «شگفت‌زده» شده‌اند؛ زیرا جیمی کارتر در مبارزات انتخاباتی خود صریحاً به سیاست حقوق بشر تأکید کرده بود و همزمان ایران توسط عفو بین‌الملل به دلیل موارد نقض حقوق بشر محکوم شده بود. حتی در صورتی هم که کارتر از گزارش عفو بین‌الملل آگاه نمی‌بود، حتماً از گزارش‌های سفارت آمریکا در تهران در مورد شکنجه‌ی مخالفان شاه مطلع بود.
\par
شاه پس از این سخنان حمایت‌آمیز، قدرت مضاعفی را برای مقابله با مخالفانش یافت. یک‌هفته بعد از سفر کارتر یعنی در تاریخ 17 دی 1356، مقاله‌ی جنجالی \textit{«ایران و استعمار سرخ و سیاه»} در روزنامه‌ی اطلاعات علیه انقلاب اسلامی به چاپ رسید. در واکنش به این مقاله، ناآرامی‌هایی در کشور رخ داد. در مشهد تظاهرات انجام شد و در روز 18دی در قم بازار و حوزه‌ی علمیه تعطیل شدند.
\par
تظاهرات مردم قم با تیراندازی نیروهای حکومت کشته داد. دانشجویان دانشکده‌ی فنی تهران نیز 2روز بعد در محوطه‌ی دانشگاه تظاهرات کردند و کلاس‌ها را تعطیل کردند و بازار تهران نیز در روز چهلم شهادت کشتگان قم تعطیل شد. بدین ترتیب سلسله راهپیمایی‌های روز چهلم شهدای شهرها آغاز شد و هر 40روز در شهری تظاهرات می‌شد.
\par
درج مقاله‌ی هتاکانه‌ی «ایران و استعمار سرخ و سیاه» در صفحه‌ی 7 روزنامه‌ی اطلاعات، مورخ 17 دی‌ماه 1356، نقطه‌ی آغازی بر پایان رژیم شاهنشاهی ایران بود. این مقاله که با نام مستعار احمد رشیدی مطلق منتشر شد و از داریوش همایون، وزیر اطلاعات و جهانگردی وقت، نشئت می‌گرفت، چنان طوفانی به راه انداخت که 400روز بعد طومار خاندان پهلوی را در هم پیچید.
\end{document}